\documentclass{resume}
\usepackage{zh_CN-Adobefonts_external} 
\usepackage{linespacing_fix}
\usepackage{cite}
\usepackage{hyperref}
\hypersetup{
    colorlinks=true,
    linkcolor=cyan,
    filecolor=magenta,      
    urlcolor=blue,
}

\begin{document}
\pagenumbering{gobble}

\MyName{张继华}
% 
\SimpleEntry{政治面貌:共产党员}
\SimpleEntry{籍贯:江苏南通}
\SimpleEntry{Tel: 15371939295}
\SimpleEntry{\href{mailto:zhangjihua0327@outlook.com}{个人邮箱:zhangjihua0327@outlook.com}}
% \SimpleEntry{\href{https://lollipopzzz.cn}{个人主页:https://lollipopzzz.cn}}
\yourphoto{0.12}

\section{教育背景}
\datedsubsection{\textbf{南京大学},软件工程,\textit{专业硕士}}{2024.9 - 2026.6}
\begin{itemize}
    \item 研究方向:分布式系统的形式化验证
    \item 荣誉:南京大学优秀共青团员(2025)
\end{itemize}
\datedsubsection{\textbf{南京大学},软件工程,\textit{本科}}{2020.9 - 2024.6}
\begin{itemize}
    \item 保研学分绩排名23\%
    \item 荣誉:连续三年人民奖学金(2021-2023),南京大学优秀招生志愿者。
\end{itemize}

\section{专业技能}
\begin{itemize}
    \item \textbf{计算机:}
    \begin{itemize}
        \item 熟悉Java, Spring相关生态;了解 Spring AI,有 Agent 开发经验;
        \item 熟悉使用数据库,缓存,消息队列等各类中间件和基础设施,了解其底层实现;
        \item 理解领域驱动设计的设计理念和架构,分布式系统,微服务架构,理解掌握各种软件工程图和文档,熟悉各种设计模式。
    \end{itemize}
    \item \textbf{英语:}{六级574,能快速阅读英文文档}
\end{itemize}

\section{实习经历}
\datedsubsection{\textbf{阿里国际},后端开发工程师}{2025.5 - 2025.10}
\Content
{{设计中心-AI设计中台}}
{
    \begin{itemize}
        \item 堆友和AI能力中台的开发和维护工作:\\ 对堆友中台原有AI能力MCP化改造,提供扩展能力,可以上架MCP平台;\\ 基于LLM构建可用的Agent工作流,向用户提供易用,具有教学或指导意义的Agent;
        \item 堆友学堂线下课程部分后端完整从0到1开发工作,开发完整的线下课程展示、售卖和运营管理模块,拓展经营模式:\\
        数据库和架构设计和实现,落地完整的业务逻辑;\\
        接入支付系统,实现支付逻辑和退款服务,基于TCC实现分布式事务;
        \item ...
    \end{itemize}
}
\datedsubsection{\textbf{小米南京分公司},后端开发工程师}{2023.6 - 2023.10}
\Content
{{互联网业务部-游戏媒体}}
{
    \begin{itemize}
        \item 主要负责游戏媒体社区项目 viewpoint 和游戏商城 game-store的开发维护工作;
        \item 基于缓存和数据库,构建统一的点赞服务,支持热点数据快速访问、分支下排序、分页查询等功能;
        \item 基于Elasticsearch的基于标签的相关帖子推荐功能的实现。
    \end{itemize}
}

\section{实践项目}
\datedsubsection{\textbf{南京大学软件学院实习就业系统},后端开发、项目管理}{2024.7 - 2024.9}
\Contents
{实际投产项目:用于实习申请流程线上化,本研学生实习信息确认和统计。}
{\begin{itemize}
    \item 基于领域驱动设计的服务器后端项目架构设计与实现,数据库设计;
    \item 构建完全独立在内网的项目开发平台和运行平台,实现CI-CD、数据库、对象存储,网络拓扑设计一系列需求,接入学校统一身份认证;
    \item 主要负责后端设计与开发,项目管理,架构和网络拓扑设计,向学工办公室老师汇报.
\end{itemize}
}

\datedsubsection{\textbf{tai-e 静态程序分析器},软件静态分析课程作业}{2022.10 - 2022.12}
\Contents
{基于Java、Soot,实现Java程序静态分析器;可以实现对Java代码的活跃变量分析,死代码分析,常量传播分析,上下文敏感和不敏感的指针分析,和污点分析等。}
{\begin{itemize}
    \item 使用Soot框架对Java程序进行静态分析,生成三地址中间代码(IR);
    \item 对三地址中间代码进行分析,生成程序的控制流图和数据流图;
    \item 对控制流图和数据流图进行分析,生成程序的静态分析报告。
\end{itemize} }
\end{document}