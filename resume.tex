\documentclass{resume}
\usepackage{zh_CN-Adobefonts_external} 
\usepackage{linespacing_fix}
\usepackage{cite}
\usepackage{hyperref}
\hypersetup{
    colorlinks=true,
    linkcolor=cyan,
    filecolor=magenta,      
    urlcolor=blue,
}

\begin{document}
\pagenumbering{gobble}

\MyName{张继华}
\sepspace
\SimpleEntry{政治面貌:共产党员}
\SimpleEntry{Tel: 15371939295}
\SimpleEntry{\href{mailto:zhangjihua0327@outlook.com}{个人邮箱:zhangjihua0327@outlook.com}}
\SimpleEntry{\href{https://lollipopzzz.cn}{个人主页:https://lollipopzzz.cn}}
\yourphoto{0.14}

\section{教育背景}
\datedsubsection{\textbf{南京大学},软件工程,\textit{专业硕士}}{2024.9 - 2026.6}
\begin{itemize}
    \item 研究方向:分布式系统的形式化验证
    \item 担任本科课程程序设计基础助教
\end{itemize}
\datedsubsection{\textbf{南京大学},软件工程,\textit{本科}}{2020.9 - 2024.6}
\begin{itemize}
    \item 选修课程:软件分析(静态分析);计算机程序的形式语言;服务端开发等
    \item 保研学分绩排名27\%
\end{itemize}
\sepspace

\section{专业技能}
\begin{itemize}
    \item \textbf{计算机:}
    \begin{itemize}
        \item 熟悉Java生态,包裹 Spring Maven等;
        \item 熟悉使用 Mysql及相关插件,Redis,消息队列等各类中间件和基础设施;
        \item 理解领域驱动设计的设计理念和架构,笑裂分布式系统,微服务架构,理解掌握各种软件工程图和文档,熟悉各种设计模式。
        \item 熟练使用 Linux的工作环境
    \end{itemize}
    \item \textbf{英语:}{六级574,能快速阅读英文文档}
\end{itemize}
\sepspace

\section{实习经历}
\datedsubsection{\textbf{阿里国际},后端开发工程师}{2025.5 - 2025.10}
\Content
{{设计中心-AI设计中台}}
{
    \begin{itemize}
        \item 堆友和AI能力中台的开发和维护工作;
        \item 堆友中台AI能力MCP化改造,预备开放到百炼平台,Agent调优,搭建服务。
        \item 堆友学堂线下部分后端完整从0到1的开发工作,包括数据库设计和架构实现,接入支付系统,实现支付逻辑和退款服务;
        \item ...
    \end{itemize}
}
\sepspace
\datedsubsection{\textbf{小米南京分公司},后端开发工程师}{2023.6 - 2023.10}
\Content
{{互联网业务部-游戏媒体}}
{
    \begin{itemize}
        \item 主要负责游戏媒体社区项目 viewpoint 和游戏商城 game-store的开发维护工作;
        \item 通过维护项目和 debug 熟悉代码架构和实现;
        \item 和师兄修改存储架构实现点赞服务,统一持久化点赞数据的存储和排序能力;
        \item 基于Elasticsearch引的基于标签的相关帖子推荐功能的实现。
    \end{itemize}
}

\section{实践项目}
\datedsubsection{\textbf{南京大学软件学院实习就业系统},后端开发、项目管理}{2024.7 - 2024.9}
\Contents
{一个服务于校内软件学院本研学生,主要功能在于学生实习信息的确认和统计,学生就业情况统计的平台,已经投入使用,持续维护和改进。}
{\begin{itemize}
    \item 利用开源镜像、docker和学校的已有基础设计,构建完全独立在内网的项目开发平台和运行平台,实现CI-CD、数据库、对象存储,网络拓扑设计一系列需求,接入学校统一身份认证;
    \item 主要负责项目管理,架构和网络拓扑设计,向学工办公室老师汇报.
\end{itemize}
}
% \datedsubsection{\textbf{基于大语言模型的分布式协议归纳不变式自动生成技术},毕业设计}{2024.01 - 2024.06}
% \Contents
% {前期使用强化学习做验证(本科毕设内容),后将大语言模型替换强化学习框架,将大语言模型加入到归纳不变式生成过程中来。
%     \\ 大语言模型需要识别分布式协议的内容,分析协议的行为(状态的转移方向)和Safety Property,给出初步的归纳不变式(Inv)。
%     \\ 检验系统验证大模型给出的归纳不变式,如果正确则返回,如果错误需要将归纳反例(cti)继续交给大模型进行识别学习,并给出一定的提示词。} 
% {\begin{itemize}
%     \item 接入大模型的输入输出,并给出合适的提示词;
%     \item 基于已有工具,设计实现验证系统;
%     \item 对结果进行验证。
% \end{itemize}}

\datedsubsection{\textbf{tai-e 静态程序分析器},软件静态分析课程作业}{2022.10 - 2022.12}
\Contents
{使用Java、Soot,对Java程序进行静态分析,实现一个简单的静态程序分析器,用于完成对Java代码的活跃变量分析,死代码分析,常量传播分析,上下文敏感和不敏感的指针分析,和污点分析等。}
{\begin{itemize}
    \item 使用Soot框架对Java程序进行静态分析,生成三地址中间代码(IR);
    \item 对三地址中间代码进行分析,生成程序的控制流图和数据流图;
    \item 对控制流图和数据流图进行分析,生成程序的静态分析结果。
\end{itemize} }
\sepspace

\section{奖励荣誉}
\datedsubsection{人民奖学金}{2021,2022,2023}
\datedsubsection{南京大学优秀招生志愿者}{2024}
\datedsubsection{南京大学优秀共青团员}{2025}

\end{document}