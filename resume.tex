\documentclass{resume}
\usepackage{zh_CN-Adobefonts_external} 
\usepackage{linespacing_fix}
\usepackage{cite}
\usepackage{hyperref}
\hypersetup{
    colorlinks=true,
    linkcolor=cyan,
    filecolor=magenta,      
    urlcolor=blue,
}

\begin{document}
\pagenumbering{gobble}

\MyName{张继华}
\sepspace
\SimpleEntry{政治面貌:共产党员}
\SimpleEntry{Tel: 15371939295}
\SimpleEntry{\href{mailto:zhangjihua0327@outlook.com}{个人邮箱:zhangjihua0327@outlook.com}}
\SimpleEntry{\href{https://lollipopzzz.cn}{个人主页:https://lollipopzzz.cn}}
\SimpleEntry{开朗,乐观积极,外向性格,乐于社交}
\yourphoto{0.14}

\section{教育背景}
\datedsubsection{\textbf{南京大学},软件工程,\textit{专业硕士}}{2024.9 - 2026.6}
\begin{itemize}
    \item 研究方向:分布式系统的形式化验证
\end{itemize}
\datedsubsection{\textbf{南京大学},软件工程,\textit{本科}}{2020.9 - 2024.6}
\begin{itemize}
    \item 选修课程:软件分析(静态分析);计算机程序的形式语言;Web前端开发;服务端开发等
    \item 保研学分绩排名27\%
\end{itemize}
\sepspace

\section{实习经历}
\datedsubsection{\textbf{小米互联网业务部(南京)},后端开发工程师}{2023.6 - 2023.10}
\Content
{{游戏媒体后端开发}}
{
    {\\主要负责游戏媒体社区项目和 app-store 在线商城的日常维护和开发工作。}
    {\\在实习前期在熟悉项目代码的同时进行日常维护和debug工作。例如:修复缓存和数据库不一致性BUG,提高系统容灾能力,保障接口的返回速度。}
    {\\在实习后期,完成一个日常需求(对于 toB 业务配置项的移库工作)和一个大版本下的完整需求点:实现帖子推荐接口,需要允许使用推荐排序和提供将来接入推荐算法的接口;引入缓存,减少数据包装次数,提高访问效率。}
}
\sepspace

\section{实践项目}
\datedsubsection{\textbf{南京大学软件学院实习就业系统},后端开发、运维}{2024.7 - 2024.9}
\Contents
{一个服务于校内软件学院本研学生,主要功能在于学生实习信息的确认和统计,学生就业情况统计的平台,已经投入使用,持续维护和改进。}
{\\
项目的难点在于:基于学校的数据安全需求,需要将所有的数据存储在校内的服务器上,包括数据库和对象存储服务;
需要接入学校的统一身份认证接口;数据库表设计;做好CI-CD准备工作。
\\主要负责网络拓扑设计、运维、项目管理和后端开发。\\使用docker将每个服务作为容器管理,使用成熟开源容器作为数据和对象存储;使用gitlab-runner进行CI-CD。 }

% \datedsubsection{\textbf{SysY 语言解析器},编译原理课程作业}{2022.10 - 2022.12}
% \Contents
% {使用Java、Antlr4、LLVM(JAVACPP),对 SysY 语言程序开发一个实现绝大多数功能的的编译器(从源码到LLVM中间代码),包括此法、语法、语义分析和中间代码生成。对计算机语言的编译过程有了初步的了解,Linux环境下的程序编程、Linux使用技巧的初步了解。}
% {\begin{itemize}
%     \item 借助Anltr4编写词法和语法规则,实现词法和语法的分析器,构建抽象语法树和字符表;
%     \item 解析字符表和抽象语法树,构建字母表,生成LLVM中间代码;
% \end{itemize} }

\datedsubsection{\textbf{tai-e 静态程序分析器},软件静态分析课程作业}{2022.10 - 2022.12}
\Contents
{使用Java、Soot,对Java程序进行静态分析,实现一个简单的静态程序分析器,用于完成对Java代码的活跃变量分析,死代码分析,常量传播分析,上下文敏感和不敏感的指针分析,和污点分析等。}
{\begin{itemize}
    \item 使用Soot框架对Java程序进行静态分析,生成三地址中间代码(IR);
    \item 对三地址中间代码进行分析,生成程序的控制流图和数据流图;
    \item 对控制流图和数据流图进行分析,生成程序的静态分析结果。
\end{itemize} }
\sepspace

\section{专业技能}
\begin{itemize}
    \item \textbf{英语:}{六级574,能快速阅读英文文档}
    \item \textbf{计算机:}
    {熟练使用Java,对其他编程语言有一定的了解;熟悉Springboot应用的开发模式,有一定的开发经验;
    熟悉使用 Mysql,Redis等各类存储基础设施;熟悉使用Linux 工作环境;
    了解分布式系统,微服务架构;理解掌握并且使用过各类型UML图和各种设计模式;}
\end{itemize}
\sepspace

\section{奖励荣誉}
\datedsubsection{人民奖学金}{2021,2022,2023}
\datedsubsection{南京大学优秀招生志愿者}{2024}

\end{document}